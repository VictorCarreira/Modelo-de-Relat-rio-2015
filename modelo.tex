%%%%%%%%%%%%%%%%%%%%%~~~RELATÓRIO ANUAL 2015~~~%%%%%%%%%%%%%%%%%%%%%%%%
%%%%%%%%%%%%%%%%%%%%%~~OBSERVATÓRIO NACIONAL~~%%%%%%%%%%%%%%%%%%%%%%%
%%%%%%%%%%%%%%%%%%%%%%%%%%%%%%%%%%%%%%%%%%%%%%%%%%%%%%%%%%%%%%
%-------------------------------------------------------------------------------------------------------------------------------------------------------%
									                         	          %MODELO DE DOCUMENTO%
%-------------------------------------------------------------------------------------------------------------------------------------------------------%
\documentclass[12pt,a4paper,final]{report}%modelo do documento tipo relatório



%-------------------------------------------------------------------------------------------------------------------------------------------------------%
											                                   %PACOTES UTILADOS%
%-------------------------------------------------------------------------------------------------------------------------------------------------------%
\usepackage[utf8x]{inputenc}
\usepackage{ucs}
\usepackage[english, brazil]{babel}
\usepackage{amsmath}
\usepackage{amsfonts}
\usepackage{amssymb}
\usepackage{makeidx}
\usepackage{graphicx}
\usepackage{lipsum} % Required to insert dummy text. To be removed otherwise
\usepackage{epstopdf}%adiciona imagens em formato eps no pdf.
\usepackage{subfigure}%cria ambientes de multifiguras
\usepackage{float}%coloca as figuras exatamente aonde você quer
\usepackage[left=2cm,right=2cm,top=2cm,bottom=2cm]{geometry}
\author{Aluno....}%colocar seu nome aqui
\title{Relatório Anual Estudantil do ano de 2015}


%--------------------------------------------------------------------------------------------------------------------------------------------------------%
										                                      	%INÍCIO DO RELATÓRIO%
%--------------------------------------------------------------------------------------------------------------------------------------------------------%

\begin{document}
\thispagestyle{empty}% retira a numeração da primeira página


\begin{figure}[H]
\subfigure{\includegraphics[scale=1]{Imagens/logoON.jpg}}
\hspace{1cm}
\textbf{RELATÓRIO 2015}
\hspace{0.5cm}
\subfigure{\includegraphics[scale=1.8]{Imagens/logoMCT.png}}
\end{figure}

\begin{center}

\hspace{60cm}%espaçamento de 60cm entre a última item.

\hspace{60cm}

\textbf{TÍTULO DO TRABALHO} 
\end{center}

\hspace{60cm}

\hspace{60cm}

\begin{flushleft}

\textbf{NOME DO ALUNO:} 

\textbf{NÍVEL:} 

\textbf{ENDEREÇO RESIDENCIAL:} 

\textbf{TELEFONES FIXO/CELUAR:} 

\textbf{E-MAIL N\~AO INSTITUCIONAL:} 

\textbf{DATA DE INÍCIO NA PÓS-GRADUAÇ\~AO:} 

\textbf{DATA DE INÍCIO DO PROJETO DE PESQUISA:} 

\textbf{NOME DO ORIENTADOR:} 

\textbf{PERÍODO A QUE SE REFERE O RELATÓRIO:} 


\end{flushleft}

\hspace{60cm}

\hspace{60cm}

\hspace{60cm}

\hspace{60cm}

\hspace{60cm}



\begin{table}[h]
\centering
\begin{tabular}{rcc}
 
Entregue ao orientador em &--------------------& ----------------------------------------------------------\\ 
                      
 & Data       & Assinatura do orientador\\
 &  &  \\
 &  &  \\
 &  &  \\
Parecer do orientador &--------------------& ----------------------------------------------------------\\
 & Data       & Assinatura do orientador\\
 &  &  \\
 &  &  \\
 &  &  \\
 Recebido na Secretaria em &--------------------& ----------------------------------------------------------\\
 & Data &Assinatura do funcionário da secretaria da pós-graduação \\
 
\end{tabular}
\end{table}

\pagebreak%Final da primeira página.

%-----------------------------------------------------------------Início da parte 2-------------------------------------------------------------------%
\begin{center}
\textbf{INFORMAÇÕES CURRICULARES}
\end{center}

\hspace{60cm}


\begin{flushleft}

\textbf{TOTAL DE CRÉDITOS CURSADOS:} ----.

\hspace{60cm}

\textbf{LISTA DE DISCIPLINAS CURSADAS E CONCEITOS OBTIDOS:}

\hspace{60cm}

\begin{table}[h]
\centering
\begin{tabular}{ccc}
 
Disciplina &Situação& Conceito\\ 
                      
Disciplina X& Aprovado   & A \\%preencher aqui com as disciplinas cursadas e seus conceitos
 ....& .... & .... \\
 ....&....  &  ....\\
 ....& .... & .... \\
 ....& .... &....  \\
... &  ...&  ....\\
...& ... & ... \\
...& .... & .... \\
... & .... & ... \\
 ...& .... &....  \\
 ...& ... &...  \\
 
\end{tabular}
\end{table}

\textbf{SITUAÇ\~AO DO ALUNO QUANTO AOS CRÉDITOS OBTIDOS NOS SEMINÁRIOS ANUAIS:}------ %Aprovado no(s) ano(s) 20XX e/ou  Reprovado no(s) ano(s) 20XX

\hspace{60cm}

\textbf{SITUAÇÃO DO ALUNO QUANTO AO EXAME DE PROFICIÊNCIA:}  %Aprovado ou Reprovado

\hspace{60cm}

\textbf{SITUAÇÃO DO ALUNO DE DOUTORADO QUANTO AO EXAME DE QUALIFICAÇÃO} ------%Aprovado ou Reprovado no  ano 20XX 

\hspace{60cm}

\textbf{LISTA DAS REUNIÕES CIENTÍFICAS EM QUE PARTICIPOU NO PERÍODO A QUE SE REFERE O RELATÓRIO, COM O TÍTULO E AUTORES DOS TRABALHOS APRESENTADOS:} ----

\hspace{60cm}

\textbf{LISTA DOS ARTIGOS PUBLICADOS, ACEITOS OU SUBMETIDOS:} -----

\hspace{60cm}

\textbf{OUTRAS ATIVIDADES RELEVANTES NO PERÍODO (PARTICIPAÇ\~AO EM TRABALHOS DE CAMPO, ESCOLAS ETC):} ----%tamanho livre

\hspace{60cm}


\end{flushleft}

\pagebreak%final da parte 2

%-----------------------------------------------------------------Início da parte 3-------------------------------------------------------------------%
%Nesta parte do relatório o estudante deve incluir o projeto de pesquisa tal como apresentado à Comissão de Pós-Graduação em                                                      															Geofísica do Observatório Nacional na época da inscrição.%
%-------------------------------------------------------------------------------------------------------------------------------------------------------%


\begin{center}
\textbf{PROJETO ORIGINAL}
\end{center}

%\include{•}Pode usar esse comando para anexar o arquivo do relatório anterior. Só precisa descomentar.

\pagebreak%final da parte 3

%-----------------------------------------------------------------Início da parte 4-------------------------------------------------------------------%
%Nesta parte do relatório o estudante deve apresentar um resumo do desenvolvimento do projeto de pesquisa um ano atrás 																										, no último relatório 
																		%(NO MÁXIMO UMA PÁGINA)
%-------------------------------------------------------------------------------------------------------------------------------------------------------%


\begin{center}
\textbf{RELATO DO DESENVOLVIMENTO DO PROJETO DE PESQUISA NO ÚLTIMO RELATÓRIO }
\end{center}






\pagebreak%final da parte 4
%-----------------------------------------------------------------Início da parte 5-------------------------------------------------------------------%
											%Esta parte do relatório deve conter obrigatoriamente os seguintes itens:

																		% I - Metodologia aplicada ou desenvolvida
																			% II - Resultados parciais já obtidos
														% III - Dificuldades encontradas e como elas estão sendo superadas
																% IV - Bibliografia utilizada no contexto do trabalho.
																
%OBS: Recomenda-se que o estudante escreva, nesta parte do relatório, textos que poderão ser aproveitados em suas teses ou dissertações. Por exemplo, aqueles alunos que realizaram trabalhos de campo ou de laboratório no último ano e estes trabalhos fazem parte de seu projeto de pesquisa;  então, estes alunos podem aproveitar para escrever sobre estes trabalhos executados. 

%Outro exemplo, aqueles alunos que estão desenvolvendo metodologias ou aplicações de metodologias existente;, então, estes alunos podem aproveitar para escrever sobre estas metodologias.

%Outro exemplo, aqueles alunos mais adiantados que já escreveram nos relatórios anteriores os itens acima; então, estes alunos podem aproveitar para escrever sobre seus resultados apresentando figuras que serão aproveitadas na Tese ou Dissertação.

%Alguns alunos também podem escrever o texto da introdução do seu trabalho. Em geral, uma introdução deve: 1) apresentar a natureza e o alcance do problema; 2) revisar a literatura;, 3) apresentar os objetivos do trabalho; 4) descrever o método de investigação; e 5) descrever os principais resultados da investigação.

																			% (TAMANHO LIVRE)
%-------------------------------------------------------------------------------------------------------------------------------------------------------%
\begin{center}
\textbf{DESCRIÇÃO DETALHADA DO TRABALHO DE PESQUISA DESENVOLVIDO NO PERÍODO  DO RELATÓRIO}
\end{center}





\pagebreak%final da parte 5
%-----------------------------------------------------------------Início da parte 6-------------------------------------------------------------------%
									%Esta parte do relatório deve conter obrigatoriamente os seguintes itens:

										% I - Atividades de pesquisa previstas para o próximo período. 
										% II - Atividades acadêmicas previstas para o próximo período.
										% III - Cronograma detalhado das atividades.
										% IV - Data prevista de conclusão do mestrado ou doutorado.
% V - (*) Se houver atraso (ou previsão de atraso)  na finalização da tese ou dissertação, justifique os motivos do atraso.
																	% (NO MÁXIMO DUAS PÁGINAS)
%-------------------------------------------------------------------------------------------------------------------------------------------------------%

\begin{center}
\textbf{PRÓXIMAS ETAPAS DO TRABALHO DE PESQUISA}
\end{center}


\bibliographystyle{}%adiciona a norma da bibliografia
\bibliography{•}%adiciona o arquivo *.bib 

\end{document}


